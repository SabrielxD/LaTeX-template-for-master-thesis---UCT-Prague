% experimental part
\subsection{Subsection 1}
\label{sec:subsection_exp_1}

\textbf{Below are few examples of using the siunitx package.}\\

The DME is a capillary tube with inner diameter approx.~\SI{50}{\micro\meter}, which is connected with the mercury reservoir. The drop time was \SI{1}{\second}, scan rate was \SI{10}{\mV/\second}. Cyclic voltammetry (CV) is a linear scan voltammetry that is characteristic with faster scan rate (from \SI{50}{\mV/\second} to \SI{100}{\mV/\second} for standard analytical working electrodes with the size about 1 mm$^2$), which makes the experiment time-dependent.\cite{Bard}  

For measuring CV before and after electrolysis, a mercury "pseudodrop" electrode was used. It consists of a platinum wire which has a small sphere at one end with diameter approx. \SI{1}{\milli\meter}. The rest of the wire is sealed into a glass tube and used as a contact. The small sphere is immersed into a saturated solution of AgNO$_3$ together with a counter electrode. Electrolysis is performed at potential \SI{-1}{\volt} for \SI{5}{\minute} and by that the sphere is covered with silver. After that the electrode is immersed into mercury and silver amalgam is formed at the surface of the sphere. For renovation of the mercury surface, the tip of the electrode is again immersed into mercury. After several tens of measurements, the sphere has to be renewed. For that, electrolysis is performed in a 10\% solution of HNO$_3$ at potential \SI{1}{\volt}; by that, the silver sphere dissolves and it can be created again.

\subsection{Subsection 2}
\label{sec:subsection_exp_2}

Equation~\ref{eq:einstein} serves as \textbf{an example of a mathematical equation}.

\begin{equation}
E = mc^2
\label{eq:einstein}
\end{equation}

\noindent
It can be written also directly inline as \(E=mc^2\).